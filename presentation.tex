\documentclass[16pt]{beamer}
\usepackage[utf8]{inputenc}
\usepackage[T1]{fontenc}
\usepackage{graphicx}
\usepackage[polish]{babel}
\usepackage{url}

\input{pygments}

\usetheme{Pittsburgh}
\usenavigationsymbolstemplate{} % turn off navigation icons

\author{Silesian Ruby Users Group\\
  \footnotesize{Wojciech Wnętrzak}}
\date{December 3, 2010}
\title{Introduction to "Chef" framework}

\begin{document}

\frame{\titlepage}

\begin{frame}
  \frametitle{What is Chef?}
  \begin{itemize}
    \item Open Source (Apache License v. 2.0)
    \item Framework
    \item Ruby
  \end{itemize}
\end{frame}

\begin{frame}
  \frametitle{Chef is young}
  \begin{itemize}
    \item Released on January 15th, 2009
  \end{itemize}
  \begin{figure}
    \includegraphics[width=0.8\linewidth]{images/young-chef}
  \end{figure}
\end{frame}

\begin{frame}
  \frametitle{Why to use Chef?}
  \begin{itemize}
    \item Only one administration guy in company?
    \item Forces order in system
    \item Existing solutions for your problems
    \item Best practices
  \end{itemize}
\end{frame}

\begin{frame}
  \frametitle{How to use Chef?}
  \begin{itemize}
    \item \emph{chef-client} + \emph{chef-server}
    \item \emph{chef-client} + Opscode Platform
    \item \emph{chef-solo}
  \end{itemize}
\end{frame}

\begin{frame}
  \frametitle{Chef Server}
  \begin{itemize}
    \item Ruby gem (\emph{chef-server})
    \item Stores cookbooks
    \item Stores information about nodes
    \item Accessbile by REST API
  \end{itemize}
\end{frame}

\begin{frame}
  \frametitle{Chef Server elements}
  \begin{itemize}
    \item CouchDB -- stores node informations
    \item SOLR -- data indexing
    \item RabbitMQ -- helps in indexing
    \item Merb -- API and web user interface
  \end{itemize}
  \pause
  \begin{center}
    \LARGE \color{red} That is lot of stuff!
  \end{center}
\end{frame}

\begin{frame}
  \frametitle{Opscode Platform}
  \begin{itemize}
    \item Free plan (upto 5 nodes)
    \item Configuration step by step
    \item Organizations and users managment
  \end{itemize}
\end{frame}

\begin{frame}
  \frametitle{Chef Client}
  \begin{itemize}
    \item Ruby gem (\emph{chef})
    \item Runs on system that we want to configure
    \item Communicates with chef server
  \end{itemize}
\end{frame}

\begin{frame}
  \frametitle{Chef Solo}
  \begin{itemize}
    \item Part of \emph{chef} gem
    \item Standalone run (without connecting to server)
    \item Uses cookbooks from local tarballs
  \end{itemize}
\end{frame}

\begin{frame}
  \frametitle{Simple Workflow}
  \begin{itemize}
    \item Write cookbook with recipe
    \item Upload it to chef server
    \item Define run list by:
    \begin{description}
      \item{---} editing node on chef server
      \item{---} passing JSON file to chef-client
    \end{description}
    \item Run chef-client on desired machine
  \end{itemize}
\end{frame}

\begin{frame}
  \frametitle{Tips}
  \begin{itemize}
    \item If using \emph{RVM}, use \emph{rvmsudo} for \emph{chef-client}
    \item Take a look at chef bootstrap\footnote{\url{http://wiki.opscode.com/display/chef/Bootstrap+Chef+RubyGems+Installation}}
  \end{itemize}
\end{frame}

% TODO: use somewhere
\begin{frame}
  \frametitle{Example Recipe\footnote{\url{https://github.com/37signals/37s_cookbooks/blob/master/redis/recipes/default.rb}}}
  \input{listing/redis_recipe}
  \label{redis_recipe}
\end{frame}

\end{document}
