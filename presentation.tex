\documentclass[16pt]{beamer}
\usepackage[utf8]{inputenc}
\usepackage[T1]{fontenc}
\usepackage{graphicx}
\usepackage[polish]{babel}
\usepackage{url}

\usetheme{Pittsburgh}

% Dark version:
\input{pygments-dark}
\usecolortheme{seahorse}
\beamertemplatesolidbackgroundcolor{black}
\setbeamercolor{normal text}{fg=white}

% Light version:
% \input{pygments-light}

\setbeamertemplate{items}[ball]
\usenavigationsymbolstemplate{} % turn off navigation icons

\author{Silesian Ruby Users Group\\
  \footnotesize{Wojciech Wnętrzak}}
\date{December 3, 2010}
\title{Introduction to "Chef" framework}

\begin{document}

\frame{\titlepage}

\begin{frame}
  \frametitle{What Is Chef?}
  \begin{itemize}
    \item Open Source (Apache License v. 2.0)
    \item Framework
    \item Ruby
    \item Infrastructure configuration management tool
  \end{itemize}
\end{frame}

\begin{frame}
  \frametitle{Chef Is Young}
  \begin{itemize}
    \item Released on January 15th, 2009
  \end{itemize}
  \begin{figure}
    \includegraphics[width=0.8\linewidth]{images/young-chef}
  \end{figure}
\end{frame}

\begin{frame}
  \frametitle{Why To Use Chef?}
  \begin{itemize}
    \item Only one administration guy in company?
    \item Forces order in system
    \item Existing solutions for your problems
    \item Best practices
  \end{itemize}
\end{frame}

\begin{frame}
  \frametitle{How To Use Chef?}
  \begin{itemize}
    \item \emph{chef-client} + \emph{chef-server}
    \item \emph{chef-client} + Opscode Platform
    \item \emph{chef-solo}
  \end{itemize}
\end{frame}

\begin{frame}
  \frametitle{Chef Server}
  \begin{itemize}
    \item Ruby gem (\emph{chef-server})
    \item Stores cookbooks
    \item Stores information about nodes
    \item Accessbile by REST API
  \end{itemize}
\end{frame}

\begin{frame}
  \frametitle{Chef Server Elements}
  \begin{itemize}
    \item CouchDB -- stores node informations
    \item SOLR -- data indexing
    \item RabbitMQ -- helps in indexing
    \item Merb -- API and web user interface
  \end{itemize}
  \pause
  \begin{center}
    \LARGE \color{red} That is lot of stuff!
  \end{center}
\end{frame}

\begin{frame}
  \frametitle{Opscode Platform}
  \begin{itemize}
    \item Free plan (upto 5 nodes)
    \item Configuration step by step
    \item Organizations and users managment
  \end{itemize}
\end{frame}

\begin{frame}
  \frametitle{Chef Client}
  \begin{itemize}
    \item Ruby gem (\emph{chef})
    \item Runs on machine that we want to configure
    \item Communicates with chef server
  \end{itemize}
\end{frame}

\begin{frame}
  \frametitle{Chef Solo}
  \begin{itemize}
    \item Part of \emph{chef} gem
    \item Standalone run (without connecting to server)
    \item Uses cookbooks from local tarballs
  \end{itemize}
\end{frame}

\begin{frame}
  \frametitle{Simple Workflow}
  \begin{itemize}
    \item Write cookbook with recipe
    \item Upload it to chef server
    \item Define run list by:
    \begin{description}
      \item{---} editing node on chef server
      \item{---} passing JSON file to chef-client
    \end{description}
    \item Run chef-client on desired machine
  \end{itemize}
\end{frame}

\begin{frame}
  \frametitle{Cookbooks}
  \begin{center}
    \LARGE ''Cookbooks for Chef are like RubyGems for Ruby''\footnote{I couldn't find author}
  \end{center}
\end{frame}

\begin{frame}
  \frametitle{Cookbook Skeleton}
  \begin{figure}
    \includegraphics[width=0.6\linewidth]{images/cookbook-skeleton}
  \end{figure}
\end{frame}

\begin{frame}
  \frametitle{Example Attributes File}
  \begin{footnotesize}
    \input{listing/postgres_attributes}
    \label{postgres_attributes}
  \end{footnotesize}
\end{frame}

\begin{frame}
  \frametitle{Example Recipe File}
  \begin{footnotesize}
    \input{listing/postgres_server_recipe_simple}
    \label{postgres_server_recipe_simple}
    \end{footnotesize}
\end{frame}

\begin{frame}
  \frametitle{Recipe Features}
  \begin{footnotesize}
    \input{listing/postgres_server_recipe}
    \label{postgres_server_recipe}
    \end{footnotesize}
\end{frame}

\begin{frame}
  \frametitle{Package Providers}
  \begin{itemize}
    \item Apt
    \item Yum
    \item MacPorts
  \end{itemize}
  \pause
  \begin{center}
    \large Many more
  \end{center}
\end{frame}

\begin{frame}
  \frametitle{Supported Systems}
  \begin{itemize}
    \item Debian
    \item Gentoo
    \item FreeBSD
    \item MacOSX
    \item Solaris
    \pause
    \item Windows
  \end{itemize}
  \pause
  \begin{center}
    \large And more
  \end{center}
\end{frame}

\begin{frame}
  \frametitle{Resources\footnote{\url{http://wiki.opscode.com/display/chef/Resources}}}
  \begin{itemize}
    \item package
    \item template
    \item file
    \item user
    \item execute
    \item script (bash, ruby, perl, python, csh)
    \item http\_request
    \item deploy
  \end{itemize}
  \pause
  \begin{center}
    \large Many more
  \end{center}
\end{frame}

\begin{frame}
  \frametitle{Additional Tools - Ohai}
  \begin{itemize}
    \item Released as a gem -- \emph{ohai}
    \item Collects system configuration/information
    \item Returns \emph{JSON}
  \end{itemize}
\end{frame}

\begin{frame}
  \frametitle{Additional Tools - Knife}
  \begin{itemize}
    \item Part of \emph{chef} gem
    \item Console tool for chef server managment
  \end{itemize}
\end{frame}

\begin{frame}
  \frametitle{Tips}
  \begin{itemize}
    \item If using \emph{RVM}, use \emph{rvmsudo} for \emph{chef-client}
    \item Take a look at chef bootstrap\footnote{\url{http://wiki.opscode.com/display/chef/Bootstrap+Chef+RubyGems+Installation}}
    \item Remember that Ruby (Chef) uses \emph{sh}, not \emph{bash}
  \end{itemize}
\end{frame}

\begin{frame}
  \frametitle{Useful Links}
  \begin{footnotesize}
    \begin{itemize}
      \item \url{http://www.opscode.com/chef/}
      \item \url{http://help.opscode.com/faqs/start/how-to-get-started}
      \item \url{http://cookbooks.opscode.com/}
      \item \url{https://github.com/opscode/cookbooks}
    \end{itemize}
  \end{footnotesize}
\end{frame}

\begin{frame}
  \frametitle{Thank You}
  \begin{center}
    \Huge Questions?
  \end{center}
\end{frame}

\end{document}
